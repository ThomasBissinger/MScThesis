\documentclass[8.5pt,twoside,twocolumn]{article}
%% CHECK: Remove all ``CHECK'' comments by solving their problems
\usepackage[utf8]{inputenc}
\usepackage{MyChem}
\usepackage{MyStandard}
\usepackage{courier}
\usepackage{booktabs}
%%\usepackage{tabto}
%% \usepackage{tabu}
\usepackage{longtable}
\usepackage{enumerate}
\usepackage{tikz}
\usetikzlibrary{patterns,calc,decorations.pathmorphing}

\makeatletter
\newcommand{\subalign}[1]{%
  \vcenter{%
    \Let@ \restore@math@cr \default@tag
    \baselineskip\fontdimen10 \scriptfont\tw@
    \advance\baselineskip\fontdimen12 \scriptfont\tw@
    \lineskip\thr@@\fontdimen8 \scriptfont\thr@@
    \lineskiplimit\lineskip
    \ialign{\hfil$\m@th\scriptstyle##$&$\m@th\scriptstyle{}##$\crcr
      #1\crcr
    }%
  }
}
\makeatother

\newcommand\snakedeko{{snake, segment length=1.5mm, amplitude=.5mm}}

\newcommand\eint{\enmat{E^{int}}}
\newcommand\eads{\enmat{E^{ads}}}
\newcommand\ezp{\enmat{E^{ZPE}}}

\renewcommand\Hil{\enmat{\mathcal H_a}}

\renewcommand\H{\enmat{\bo H}}

\newcommand\di{\te{d}}
\newcommand\dr{\di\r}
\newcommand\drr{\di\r\ }
\renewcommand\ij{_{ij}}
\newcommand\rms{\te{RMS}}
\newcommand\apd{\te{APD}}
\renewcommand\K{{\enmat{~\te{K}}}}
\renewcommand\r{\bo r}
\newcommand\ri{\enmat{\r_i}}
\newcommand\rip{\enmat{\r_{i'}}}
\newcommand\indr{\enmat{\int\di \r}}

\newcommand\id{\hskip.2cm}
\newcommand\pab{\enmat{\phi_{AB}}}



\newtheoremstyle{standard}{10 pt}{10 pt}{\hangindent=0.25in\itshape}{}{\normalfont\bfseries}{}{\newline}{}
\theoremstyle{standard}
\newtheorem{theo}{Theorem}
\newtheorem{lem}[theo]{Lemma}
\newtheorem{res}[theo]{Result}
\newtheorem{defi}[theo]{Definition}
\newtheorem{exm}[theo]{Example}
\newtheorem{cor}[theo]{Corollary}

%% FROM http://tex.stackexchange.com/questions/159139/what-is-required-to-use-background-layer-as-specified-in-tikz-manual
\pgfdeclarelayer{background}
\pgfdeclarelayer{foreground}
\pgfsetlayers{background,main,foreground}  

%% GRAPH COMMANDS
\newcommand\fp[2]{\draw {#1} node[circle,minimum size=0.2cm,draw,fill=black]
({#2}) {};} %field point
 \newcommand\lp[2]{\draw {#1} node[circle,minimum size=0.2cm,draw,fill=white]
({#2}) {};} % labelled point
\newcommand\flp[2]{\draw {#1} node[circle,minimum
size=0.2cm,draw,dashed,fill=white] ({#2}) {};} %free labelled point
\newcommand\connect[1]{
\begin{pgfonlayer}{background}
 \foreach \x/\y in {#1} {\draw (\x.center) -- (\y.center);}
\end{pgfonlayer}
}

\newcommand{\itemEq}[1]{%
        \begingroup%
        \setlength{\abovedisplayskip}{0pt}%
        \setlength{\belowdisplayskip}{0pt}%
        \parbox[c]{\linewidth}{\begin{flalign}#1&&\end{flalign}}%
        \endgroup}


\parindent0pt

\linespread{1.5}
\title{Surface Adsorption on Interstellar Ice $I_h$}
\linespread{1}
% \subtitle{Universit�t Stuttgart, Prop�deutikum zur Bachelorarbeit}
\author{T. Bissinger}

\begin{document}


\begin{titlepage}

\begin{center}

% Oberer Teil der Titelseite:

 

% Title

\newcommand{\HRule}{\rule{\linewidth}{0.5mm}}

\HRule \\[0.4cm]

{ \huge \bfseries Surface Adsorption on Interstellar Ice $I_h$}


\HRule \\[2cm]

{\LARGE Thomas Bissinger}\\[4cm]

\vfill 
%% CHECK: Stuttgart Figure
[width=7cm]{./Figures/unistuttgart.jpg}\\[2cm]   

{\Large Master Thesis supervised by \\[.7cm]
Prof. Dr. rer. nat. J. Kästner \\[.4cm]
M. Sc. Jan Meisner}\\[.4cm]


{ \Large  Institute for Theoretical Chemistry, Universität Stuttgart, December 2015}


% Author and supervisor

% Unterer Teil der Seite

\end{center}


\end{titlepage}
\newpage
\newpage
\clearpage



% \tableofcontents
% \newpage
\twocolumn[
	\maketitle
  \begin{@twocolumnfalse}
    \maketitle
    \begin{abstract}
      This abstract explains what happens. Benchmark, adsorption, maybe something about the interstellar playground.
\\
\\
    \end{abstract}
  \end{@twocolumnfalse}
]
\section{Introduction}
\label{Sec:Intro}
Interstellar chemistry is the key ingredient to understanding the molecular abundancies in our universe. While
the formation of atoms takes place in stars \cite{AtomsInStars}, their further reaction and therefore the composition
of molecular compunds in space largely happens in interstellar clouds. With modern telescopes it is possible to
measure the molecular abundancies in the interstellar medium (ISM) to a high degree of accuracy. It became
evident that the reaction rates governing the formation of molecules can not properly be explained by gas phase
chemistry alone. A prominent theory to mend this discrepancy is to consider the contribution of surface reaction
on interstellar dust grains. The dust has been measured \cite{DustMeasure}, so the question is not so much if the
reaction happens but rather to quantify its effect.

At temperatures as low as in cold interstellar clouds (between 10 and 100 $K$), the grains are covered in ices,
most prominently $\hto $ and $\co$. In the case of water, one faces \ep{amorphous solid water} (ASW), but
at very low temperatures one may find a state close to the crystalline $I_h$ state of frozen water.

Various works have already considered surface reactions.

(Here we will cite some experimentalists).

The research described above typically chose a model to describe the processes taking place in the experiment and
adjusted simulation data -- including parameters like the diffusion coefficient -- to fit the experimental data.
That would mean: if surface reaction processes are the key to filling the gap between observed and predicted
reaction rates, the input parameters might be good approximations to the actual coefficients. This is a legit
approach since one can not easily think of other processes than surface and gas phase reactions to lead to
molecular formation.

%% CHECK: Andere Parameter als Diffusionskoeffizient?
However, there is not yet a proper \ep{ab initio} theoretical calculation of surface diffusion. This means that
evaluating the parameters found in simulation is a very difficult task -- there is neither a recommended value nor 
are there any error bars to such a value. This work tries to make a first step toward the accurate simulation of surface
adsorption, diffusion and reaction of small molecules on interstellar ices. Its aim is to establish a model
of crystalline $I_h$ water in which simulations can be carried out. The main mathematical tool for describing the 
chemistry of this surface is a subdomain treated by \ep{density functional theory} (DFT) within a bigger domain
where the interaction is modelled by \ep{molecular mechanics} (MM). The two domains are coupled by a QM/MM
coupling scheme.

We will describe the theory underlying the model in the next section. Section \ref{Sec:Bench} then describes the
benchmarking we performed on smaller test systems to determine the best functionals and basis sets to use for 
the actual system. After that, we give our results for adsorption energies in Section \ref{Sec:Ads} and finally 
there will be concluding remarks and an outlook on possible further application for our findings in Section 
\ref{Sec:Con}.

\section{Theoretical Background}
\label{Sec:Theo}
This section focuses on the theoretical framework of the ice surface model. We introduce the main chemical
nomenclature in Section \ref{Sec:Theo:Interaction} and then proceed to the mathematical ideas behind
DFT in Section \ref{Sec:Theo:QMMet}. After that, Section \ref{Sec:Theo:QMMM} will explain how we describe
the MM interaction of the system and how QM and MM are coupled by the QM/MM procedure.

\newcommand\X{\enmat{\te X}}
\newcommand\Y{\enmat{\te Y}}
\newcommand\XY{{\enmat{\X - \Y}}}
\renewcommand\S{\enmat{\te S}}
\newcommand\sX{\enmat{\te {s-X}}}
\newcommand\A{\enmat{\te A}}
\subsection{Different Types of Energy}
\label{Sec:Theo:Interaction}
We will consider the \ep{interaction energy} between two molecular species X and Y. We call
the system of both molecules $\XY$. We also consider the \ep{adsorption energy} of a molecule X on
the ice surface S. We call this system $\sX$. If not declared elsewise, all appearing energies are
electronic ground state energies. We wil describe the interaction energy first.

Consider a system of two molecules X and Y. We can calculate the energy of the isolated molecule X to
be $E_{\te X}$ and the energy of the isolated molecule Y to be $E_{\te Y}$. We can also calculate the 
energy of the full system $\XY$, which will in general depend on the distance and the orientation
of the two molecules, to be $E_{\XY}$. Then, the interaction energy between the two molecules
is the energy given by
\begin{equation}
 \eint_{\XY} := E_{\XY} - E_\X - E_\Y.
 \label{Theo:InteractionEnergy}
\end{equation}
We did not include spatial dependency of $E_{\XY}$ into the above definition. A map
\mbox{$(R_\XY, \Om_\X, \Om_\Y) \mapsto \eint_\XY$} with the center of mass separation
$R_\XY$ and the molecular orientation $\Om_\X$ and $\Om_\Y$ is called the \ep{potential energy surface} (PES)
of the intermolecular interaction. It may also contain internal deformations of the molecule.

However, one does often speak of the interaction energy of two molecules without further specification
of a point on the PES. This is usually a reference to the minimum geometry
of $\XY$, that is the global minimum of $E_{\XY}$ and therefore $\eint_\XY$. If the interaction between $\X$ and $\Y$
were purely repulsive, that is $\eint_{\XY} > 0$ for all geometries, the global minimum would
not be well-defined since it requires inifinite separation of $\X$ and $\Y$ in an arbitrary direction.
However, the algorithms we use will converge to a local minimum around the initial geometry we
specify, by which we will then classify the strength of the repulsion. Even for attractive potentials,
that is potentials where $\eint_\XY < 0$ for some geometries, we are not able to determine whether 
the potential energy minimum we find is the global minimum or within what error its energy is
to the global minimum.

The adsorption energy is mostly similar. There, we have the energy of the isolated species
$E_\X$ and the energy minimum of the surface $E_{\S}$. If we denote the system of the surface 
with the adsorbed molecule $X$ by $\sX$ and its energy minimum by $E_\sX$, we define
the adsorption energy to be
\begin{equation}
 \eads_{\sX}:= E_{\sX} - E_\X - E_\S.
 \label{Theo:AdsorptionEnergy}
\end{equation}
Again, we did not include the dependency of $E_\sX$ and $E_\S$ on the respective geometries. We
even specified that we will consider the individual geometry of minimum energy here. This is quite sensible
because the surface geometry of the system $\sX$ \mbox{(surface + molecule)} may be different from the
system $\S$ of the surface alone when comparing energy minima. For a fixed value of $E_\S$, 
we could again consider a PES of the type \mbox{$(\bo r_i)_i \mapsto \eads_\sX$}, where the
vector $\r_i$ is the coordinate of the $i$-th atom in $\sX$, $1 \le i \le N$ for some $N$. The different energy minima of 
this map to $\eads_\sX$ are called \ep{binding geometries}, and the position of the
center of mass of the molecule $\X$ in a binding geometry is called a \ep{binding site}.
Exploring binding sites and the strength of the binding $\eads_\sX$ can be useful
to simulation.

The superscripts on $\eint$ and $\eads$ may be ignored if it is sufficiently clear which energy 
is meant.

We also want to introduce a third type of energy, the \ep{zero-point (vibrational) energy}.
We describe it for some general system $\A$ that may contain any arrangement of atoms.
For all calculations we do, we will work with fixed values for the atomic coordinates
$\r_i$, $1 \le i \le N$ for some $N \in \N$. That description can only be accurate if
the atoms were classical particles. However, if we want to allow for them to be
quantum objects, we need to include uncertainty into their position. This is usually
done by including the zero-point vibrational energy of the atoms. It is computed from the
\ep{Hessian} matrix $\H$ of the system,
\begin{equation}
 \H(\r_1,\ldots,\r_N) := \rb{\deri {^2 E} {\r_i \del \r_j} (\r_1,...,\r_N) }_{1\le i,j \le N}
 \label{Theo:Hessian}
\end{equation}
by
%% CHECK: Is this correct?
\begin{equation}
 \ezp=E + \frac 1 2 \te {tr} \ed{\H} = E + \Dl \ezp,
 \label{Theo:EZP}
\end{equation}
where $\te {tr} \ed{\H}$ is the \ep{trace} of the matrix $\H$, that is the sum of the eigenvalues
of $\H$. In the harmonic approximation, which should be accurate for the vibrational ground 
state, the eigenvalues of $\H$ are 
%% CHECK: proportional or identical?
proportional
to the eigenfrequencies of the system. We call the term $\Dl \ezp$ the \ep{zero-point (vibratinal energy)
correction}.

We will always use the superscript in $\ezp$ if we want to denote energies that are corrected 
with $\Dl \ezp$ in \eqref{Theo:EZP}.

\subsection{Methods of Quantum Chemistry}
\label{Sec:Theo:QMMet}

We already saw a few different energy expressions so far. The accurate calculation of these 
is naturally vital to anything we want to do in this work. We now want to focus on the methods
of quantum chemistry which will be used to describe the quantum mechanical part of our system.

For a system with a time-independent potential $V$, one usually looks for solution of the 
\ep{time-independent Schrödinger equation}
\begin{equation}
 \Ha \keP{} = E \keP{}.
 \label{Theo:Schroedinger}
\end{equation}
This equation holds for all non-relativistic quantum mechanical particles. Within the \ep{Born-Oppenheimer}
approximation, one can separate the dynamics of the atomic cores from the dynamics of the electrons. We will treat
the cores in a more or less classical way, therefore we will for now focus on solving the Schrödinger 
equation for $N < \infty$ electrons, that is the Hamiltonian of our system is
\begin{equation}
 \Ha = -\sum_{i=1}^N \frac{\hbar^2}{2m_e} \nabla_i^2 + \frac {e^2} {4 \pi \e_0} \sum_{1\le i < j \le N} \frac 1 {\abs{\r_i - \r_j}} + V(\r^N).
 \label{Theo:Hamiltonian}
\end{equation}
$\r_i$ is the space coordinate of electron $i$ and $\nabla_i^2$ is the \ep{Laplace operator} applied to the three coordinates contained in $\r_i$. 
The electrons move in an external potential $V$ given by the core geometry and movement, where $\r^N$ is the vector containing
all electron coordinates. We will have for $K < \infty$ atomic cores 
\begin{equation}
\begin{aligned}
V(\r^N)=&-\sum_{A=1}^K \frac{\hbar^2}{2m_A} \nabla_A^2 + \frac {e^2} {4 \pi \e_0} \sum_{A=1}^K \sum_{B=1}^K \frac {Z_A Z_B} {\abs{\bo R_A - \bo R_B}} \\
   &- \frac {e^2} {4 \pi \e_0} \sum_{A=1}^K \sum_{i=1}^N \frac {Z_A} {\abs{\bo R_A - \r_i}}. 
\end{aligned}
\label{Theo:CorePotential}
\end{equation}
Here, $Z_A$ is the atomic number of atom $A$ and $\bo R_A$ is the coordinate of it with corresponding $\nabla_A^2$. We
can separate $V$ into a core-core (cc) and an core-electron (ce) potential
\begin{equation}
 V(\r^N)=V_{cc}+V_{ce}(\r^N)
\label{Theo:SeparatePotential}
\end{equation}
with
\begin{equation}
 V_{ce}(\r^N)=\sum_{i=1}^N \tilde V(\r_i) = - \frac {e^2} {4 \pi \e_0} \sum_{A=1}^K \sum_{i=1}^N \frac {Z_A} {\abs{\bo R_A - \r_i}}.
 \label{Theo:CoreElectronPotential}
\end{equation}



There will be more than one solution to \eqref{Theo:Schroedinger}, so one can construct the set of all solutions $\st{\keP{_i} | i \in \N_0}$
with corresponding energy eigenvalues $E_i$. $\st{\keP{_i}}$ always is the complete basis of
some $\mathbb C$-vector space $\Hil$, where the subscript $a$ denotes antisymmetry according 
to the \ep{Pauli principle}
\begin{equation}
\begin{aligned}
 &\bkt{\x_1,\ldots,\x_l,\ldots,\x_k,\ldots,\x_N | \Psi_i } \\
 &=\ph - \Psi_i(\x_1,\ldots,\x_l,\ldots,\x_k,\ldots,\x_N) \\
 &= - \Psi_i(\x_1,\ldots,\x_k,\ldots,\x_l,\ldots,\x_N),
\end{aligned}
\label{Theo:Pauli}
\end{equation}
where we used $\x_i=(\r_i,s_i)$ for orbital coordinates $\r_i$ and the spin coordinate $s_i$.

 If $\Hil \suseq \mathcal {L}^2$, which is usually the case,
the set of solutions can be chosen orthonormal $\bkt{\Psi_i | \Psi_j} = \dl_{ij}$.
The set of solutions is usually not finite and the set of eigenvalues (energies) $E_i$ of $\Ha$
is not necessarily bounded from above. But there is always a minimum energy, denoted by $E_0$, 
which we call the \ep{(electronic) ground state energy}. The corresponding eigenvector 
$\keP{_0}$ is the \ep{(electronic) ground state}. They can both be obtained by the
variational ansatz
\begin{equation}
\begin{aligned}
 E{_0}&= \min_{\keP{}} \st{ \brP{} \Ha \keP{}}, \\
 \keP{_0}&= \argmin_{\keP{}} \st{ \brP{} \Ha \keP{}}.
\end{aligned}
\label{Theo:Variation}
\end{equation}
Note that while $E_0$ is unique, there may be multiple possibilities for $\keP{_0}$, although
we will not consider that case.

Now, equation \eqref{Theo:Schroedinger} has a variety of equivalent counterparts. One of them
is the key to the approach of DFT. When multiplying \eqref{Theo:Schroedinger} by the bra $\brP{}$,
one can interpret the resulting energy as a (non-linear) functional of the wavefunction by
\begin{equation}
 E : \Hil \rar \R, \qquad E\ed{\keP{}}=\brP{} \Ha \keP{},
 \label{Theo:WaveFunctional}
\end{equation}
which would mean that the ground state energy can be found by minimizing the functional
$E\ed{\keP{}}$ according to \eqref{Theo:Variation}. But so far, the minimization
of said functional only differs in semantics from the task of minimizing the energy expectation value.

A truly new task arises from considering the \ep{electron density} $\rho$ instead of the wave function
$\Psi$. The two approaches are related, since the ground state electron density is given by
\begin{equation}
 \rho(\x_1,...,\x_N)=\abs{\Psi_0(\x_1,...,\x_N)}^2,
 \label{Theo:NElectronDensity}
\end{equation}
This $N$-electron density describes the probability of finding the system in a state within 
a small volume of $\di \x_1 \cdots \di \x_N$ around $(\x_1,\ldots,\x_N)$. 
The $N$-electron density can be reduced to the one-electron density by
\begin{equation}
 \rho(\r_1,s_1)=\int\di\x_2 \cdots \int\di \x_N \abs{\Psi_0(\x_1,...,\x_N)}^2,
 %\rho(\r_1)=\int\di s_1\int\di\x_2 \cdots \int\di \x_N \abs{\Psi_0(\x_1,...,\x_N)}^2,
 \label{Theo:1ElectronDensity}
\end{equation}
where the integrals run over %the spin of the first particle and
the full spin-orbit space for all particles but the first.

For a system of $N$ electrons, Hohenberg and Kohn \cite{HohenbergKohn} were able to show that the ground state
one-electron density uniquely determines the Hamiltonian except for the addition of a constant,
and that conversely there is a functional of the density that has its minimal value at the one-electron
ground state density, and for which the minimum value is the ground state energy. Therefore, the task
of solving the Schrödinger equation \eqref{Theo:Schroedinger} is reduced to the task of finding
the minimum of this density functional.

The problem here is that not much is known about the nature of this density functional. The
popular ansatz by Kohn and Sham \cite{KohnSham} is 
% CHECK: Role of V_{cc}
\begin{equation}
\begin{aligned}
 E\ed{\rho}=&V_{cc}+\indr\ \tilde V(\r) \rho(\r) + \frac {e^2}{2}\indr\indr' \frac {\rho(\r)\rho(\r')}{\abs{\r-\r'}}\\
 &+T_s 	\ed{\rho}+E_{xc}\ed{\rho}
\end{aligned}
\end{equation}





\subsection{Molecular Mechanics and QM/MM}
\label{Sec:Theo:QMMM}

\newpage
%\subsection{The Fletcher Surface}
%\label{Sec:Theo:Fletcher}

\section{Benchmarking}
\label{Sec:Bench}

\section{Adsorption on the Ice Surface}
\label{Sec:Ads}

\subsection{The Surface Model}
\label{Sec:Ads:Model}

\subsection{Results}
\label{Sec:Ads:Results}

\section{Conclusion}
\label{Sec:Con}

\section*{Acknowledgment}
I want to thank Professor J. Kaestner for his support and Mr. Jan Meisner for
his introduction to the topic and his company along the way.


%\clearpage
% \bibliographystyle{apalikenotitle}
%\bibliographystyle{ieeetr}
% \begin{thebibliography}
    %\bibliography{BibForsch2}

\end{document}

