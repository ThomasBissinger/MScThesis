\documentclass[crop,tikz]{standalone}
%\documentclass{article}
\usepackage{tikz}
\usepackage{tikz-3dplot}
\usepackage{MyStandard}
\usetikzlibrary{patterns,calc}
%\usepackage{tikz-3dplot}

\begin{document}
\newcommand\PI{3.14}
\newcommand\cycles{6.5}
\newcommand\bottom{-.5}
\def\shift{7.5}
\def\shiftbelow{-5}
\newcommand\nextarrow{\draw[->,line width=2pt] (1,.5) -- (2,.5);}
\newcommand\imgtitle[1]{\node[anchor=west] (Name) at (-5,2.5) {\large{\textbf{#1}}};}
\newcommand\imgcaption[1]{\node[anchor=north west,text width=5cm] (Caption) at (-5.1,-.5) {\small{#1}};}
%\newcommand\surface[3]{\draw[domain=#1:#2,smooth,variable=\x,blue] plot ({\x}, {#3*sin((\x-#1)/(#2-#1)*360*\cycles});}
\newcommand\surface[4]{\def\stdlen{#2-#1}
\def\period{\stdlen/\cycles/.9}
\draw[pattern=horizontal lines, pattern color=gray] (#1,\bottom) -- plot [samples=150, domain=#1:#2,smooth,variable=\x,blue] ({\x}, {#3*sin((\x-#1)/(#2-#1)*360*\cycles}) -- (#2,\bottom) -- cycle;
\foreach \x in {0,1,2,3,4,5}
\coordinate (#4 Site\x) at ({#1+.1 + (\period * (\x))},.2);}
%\coordinate (#4 Site\x) at ($(#1,0)+({\period * (\x-1.6)},0)$);}
\newcommand\atom[2]{\shadedraw[shading=radial,outer color=#2!90,inner color=#2!30,draw=black] (#1) circle(.3cm);}
\newcommand\rightof[1]{($(#1)+(.3,0)$)}
\newcommand\leftof[1]{($(#1)-(.3,0)$)}
\newcommand\aboveof[1]{($(#1)+(0,.3)$)}
\newcommand\belowof[1]{($(#1)-(0,.3)$)}
\newcommand\bentedge[2]{edge[dashed,line width=1.5pt,#1,color=#2]}

\begin{tikzpicture}
\begin{scope}
\imgtitle{Langmuir--Hinshelwood (LH) mechanism}
\surface{-5}{.5}{.2}{Surf1}
\coordinate (A1) at (-4.5,1.3);
\coordinate (A2) at (0,1.3);
\atom{A2}{blue}
\atom{A1}{red}
\draw \rightof{A1} \bentedge{out=0,in=90,->}{red} (Surf1 Site1);
\draw \leftof{A2} \bentedge{out=180,in=90,->}{blue} (Surf1 Site4);
\draw[line width=1pt] (0,-1.5) -- (10,-1.5);
\nextarrow
\imgcaption{(i). adsorption}
\end{scope}
\begin{scope}[shift={(\shift,0)}]
\surface{-5}{.5}{.2}{Surf1}
\coordinate (A1) at (Surf1 Site1);
\coordinate (A2) at (Surf1 Site4);
\atom{A2}{blue}
\atom{A1}{red}
\draw \aboveof{A1} \bentedge{bend left=90,bend right=-70,->}{red} ($(Surf1 Site2)-(.1,0)$);
\draw \aboveof{A2} \bentedge{bend right=90,bend left=-70,->}{blue} ($(Surf1 Site3)+(.1,0)$);

\draw($(Surf1 Site3)-(.1,0)$)  \bentedge{out=0,in=90}{blue} ($(Surf1 Site2)!.5!(Surf1 Site3)+(0,.4)$);
\draw ($(Surf1 Site2)!.5!(Surf1 Site3)+(0,.4)$) \bentedge{out=180,in=90,->}{blue} ($(Surf1 Site2)+(.1,0)$);
\nextarrow
\imgcaption{(ii). diffusion}
\end{scope}

\begin{scope}[shift={(2*\shift,0)}]
\surface{-5}{.5}{.2}{Surf1}
\coordinate (A3) at ($(Surf1 Site2)+(1.5,1.3)$);
\atom{A3}{green}
\draw \leftof{A3} \bentedge{out=180,in=90,<-}{green} (Surf1 Site2);
\imgcaption{(iii). reaction (and desorption)}
\end{scope}

\begin{scope}[shift={(0,\shiftbelow)}]
\imgtitle{Eley--Rideal (ER) mechanism}
\surface{-5}{.5}{.2}{Surf1}
\coordinate (A1) at (-4.5,1.3);
\coordinate (A2) at (0,1.3);
%\atom{A2}{blue}
\atom{A1}{red}
\draw \rightof{A1} \bentedge{out=0,in=90,->}{red} (Surf1 Site1);
%\draw \leftof{A2} \bentedge{out=180,in=90,->}{blue} ($(Surf1 Site2)+(.1,0)$);
\nextarrow
\imgcaption{(i). adsorption of first reactant}
\end{scope}
\begin{scope}[shift={(\shift,\shiftbelow)}]
\surface{-5}{.5}{.2}{Surf1}
\coordinate (A1) at (Surf1 Site1);
\coordinate (A2) at (-1,1.3);
\atom{A2}{blue}
\atom{A1}{red}
\draw \aboveof{A1} \bentedge{bend left=90,bend right=-70,->}{red} ($(Surf1 Site2)-(.1,0)$);
\draw \leftof{A2} \bentedge{out=180,in=90,->}{blue} ($(Surf1 Site2)+(.1,0)$);
\nextarrow
%\imgcaption{\begin{tabular}{ll} (ii). &second reactant approaches \\ &from the gas phase \end{tabular}}
\imgcaption{(ii). second reactant approaches \phantom{(ii). }from the gas phase}
\end{scope}

\begin{scope}[shift={(2*\shift,\shiftbelow)}]
\surface{-5}{.5}{.2}{Surf1}
\coordinate (A3) at ($(Surf1 Site2)+(1.5,1.3)$);
\atom{A3}{green}
\draw \leftof{A3} \bentedge{out=180,in=90,<-}{green} (Surf1 Site2);
\imgcaption{(iii). reaction (and desorption)}
\end{scope}

\end{tikzpicture}
\end{document}
