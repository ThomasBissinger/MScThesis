% AUTHOR: David Kreplin 
  % Für generieren eines pdf:
%
% latex latex_sample_with_image
% latex latex_sample_with_image
% latex latex_sample_with_image
% dvips -Ppdf -ta4 latex_sample_with_image.dvi
% ps2pdf latex_sample_with_image.ps

%\documentclass[10pt,english,twoside,twocolumn]{article}
%\documentclass[a4paper,11pt,english]{article}
%\usepackage{a4wide}
\documentclass[11pt,twoside]{article}

\usepackage{etex}
\usepackage[utf8]{inputenc}
%\usepackage[ngerman,english]{babel}
\usepackage[ngerman,english]{babel}
\usepackage{MyChem}
\usepackage{MyStandard}
\usepackage{courier}
\usepackage{booktabs}
\usepackage{graphicx}
%%\usepackage{tabto}
%% \usepackage{tabu}
\usepackage{longtable}
\usepackage{enumerate}
\usepackage[symbol]{footmisc}
\usepackage{gensymb} % enables degree sign
\usepackage{tabularx}
\usepackage{multirow}
\usepackage{threeparttable}
\setcitestyle{square}
\usepackage[justification=justified,singlelinecheck=false,
 aboveskip=12pt,belowskip=4pt]{caption}
 \captionsetup[table]{name=Table}
 \captionsetup[figure]{name=Figure}
\usepackage{tikz}
\usepackage{pgfplots}
\usetikzlibrary{pgfplots.groupplots}
\usepackage{grffile}
\usepackage{rotating}
\usepackage{verbatim}
\usetikzlibrary{patterns,calc,decorations.pathmorphing,angles,quotes,external}
\tikzexternalize[prefix=tikz,shell escape=-enable-write18]
\usepackage{tikz-3dplot}
\usepackage{adjustbox}

%% David's packages
\usepackage{times} %,mathptm}
%\usepackage{babel}
%\usepackage{dina4}
\usepackage{psfrag}
\usepackage{amsmath}
\usepackage{amsfonts}
\usepackage{geometry}
\usepackage{float}
%\usepackage{SIunits}
%\usepackage{ziffer}
% \geometry{
% 	paper=a4paper, % Change to letterpaper for US letter
% 	inner=2.5cm, % Inner margin
% 	outer=3.8cm, % Outer margin
% 	bindingoffset=2cm, % Binding offset
% 	top=1.5cm, % Top margin
% 	bottom=1.5cm, % Bottom margin
% 	%showframe,% show how the type block is set on the page
% }
\geometry{a4paper,left=30mm,right=30mm, top=30mm, bottom=30 mm}
\usepackage{mathtools}
\usepackage{relsize}
%\usepackage[onehalfspacing]{setspace}
\usepackage{setspace}
%Makros:
\usepackage{tabu}
\usepackage[hidelinks]{hyperref}
\usepackage{fancyhdr}
\usepackage{listings}
\usepackage{fancyvrb}
\usepackage{nicefrac}
%\VerbatimFootnotes
\usetikzlibrary{plotmarks}

\numberwithin{equation}{section}	

\pagestyle{fancy}
%\pagestyle{headings}
\fancyhead{}
\fancyfoot{}
\renewcommand{\sectionmark}[1]{
\markboth{\thesection{}. #1}{}
}
\renewcommand{\subsectionmark}[1]{
\markright{\thesubsection{} #1}
}

\fancyhead[LE]{\thepage }
%\fancyhead[RE]{\thesection.\nouppercase{\leftmark }}
\fancyhead[RE]{\leftmark}
\fancyhead[LO]{ \rightmark }
\fancyhead[RO]{\thepage}

\lstset{
  basicstyle=\ttfamily\footnotesize,
  columns=fullflexible,
  keepspaces=true,
  mathescape
}
%\linespread{5}
%\renewcommand{\baselinestretch}{2.5} 
%% Make stretch:
% http://texwelt.de/wissen/fragen/3/wie-stelle-ich-einen-zeilenabstand-von-15-ein

% \usepackage{setspace}
\makeatletter
\newcommand{\MSonehalfspacing}{%
  \setstretch{1.44}%  default
  \ifcase \@ptsize \relax % 10pt
    \setstretch {1.448}%
  \or % 11pt
    \setstretch {1.399}%
  \or % 12pt
    \setstretch {1.433}%
  \fi
}
\newcommand{\MSdoublespacing}{%
  \setstretch {1.92}%  default
  \ifcase \@ptsize \relax % 10pt
    \setstretch {1.936}%
  \or % 11pt
    \setstretch {1.866}%
  \or % 12pt
    \setstretch {1.902}%
  \fi
}
\makeatother
\makeatletter
\newcommand\subsubsubsection{\@startsection{paragraph}{4}{\z@}%
            {-2.5ex\@plus -1ex \@minus -.25ex}%
            {1.25ex \@plus .25ex}%
            {\normalfont\normalsize\bfseries}}
\makeatother
\setcounter{secnumdepth}{4} % how many sectioning levels to assign numbers to
\setcounter{tocdepth}{4}    % how many sectioning levels to show in ToC
\renewcommand{\bibname}{References}
\addto\captionsenglish{% Replace "english" with the language you use
  \renewcommand{\contentsname}%
    {Contents}%
}
\addto\captionsgerman{% Replace "english" with the language you use
  \renewcommand{\contentsname}%
    {Contents}%
}

%\renewcommand{\contentsname}{Contents}
\MSonehalfspacing